\documentclass[12pt]{book}
\usepackage[cp1251]{inputenc}
\usepackage[russian]{babel}
\begin{document}
\begin{center}
 \section*{\huge Derivative 2016 } 
\end{center}

We can see that
\[
f(x) = {\left( \frac{\tg \left( x \right) }{x }\right) }^{\left( \sin \left( x \right) \right) }
\]
	
It's obvious that
\[
f'(x) = {\left( \frac{\tg \left( x \right) }{x }\right) }^{\left( \sin \left( x \right) \right) }* \left( \cos \left( x \right) * 1 * \ln \left( \frac{\tg \left( x \right) }{x }\right) + \sin \left( x \right) * \frac{\frac{\frac{1 }{\left( \cos \left( x \right) \right) ^2 }* x - 1 * \tg \left( x \right) }{x ^2 }}{\frac{\tg \left( x \right) }{x }}\right) 
\]
Most functions that occur in practice have derivatives at all points or at almost every point. Early in the history of calculus, many mathematicians assumed that a continuous function was differentiable at most points. Under mild conditions, for example if the function is a monotone function or a Lipschitz function, this is true. 
However, in 1872 Weierstrass found the first example of a function that is continuous everywhere but differentiable nowhere.
\[
f'(x) = {\left( \frac{\tg \left( x \right) }{x }\right) }^{\left( \sin \left( x \right) \right) }* \left( \cos \left( x \right) * \ln \left( \frac{\tg \left( x \right) }{x }\right) + \sin \left( x \right) * \frac{\frac{\frac{1 }{\left( \cos \left( x \right) \right) ^2 }* x - \tg \left( x \right) }{x ^2 }}{\frac{\tg \left( x \right) }{x }}\right) 
\]
\end{document}
